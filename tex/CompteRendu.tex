\documentclass[french, 10pt, a4paper]{article}
\usepackage[T1]{fontenc}
\usepackage[utf8]{inputenc}
\usepackage{lmodern}
\renewcommand{\familydefault}{\sfdefault}
\usepackage[margin=2cm]{geometry}
\usepackage{babel}

\setlength{\parindent}{0pt}

\usepackage{minted}
\setminted[ocaml]{
	tabsize=2,
	breaklines,
}

\usepackage{hyperref}
\hypersetup{
	colorlinks=true,
	linkcolor=blue,
	urlcolor=cyan,
	pdfauthor=Mathias Rivet,
}

\usepackage{array}

\title{Compte-rendu du projet Comment voter ?}
\author{
	Mathias Rivet\\
	Léo Grange\\
	Arthur Lepley
}



\begin{document}
	\maketitle

	Les codes sources sont aussi disponible à l'adresse : \href{https://github.com/programindfr/INF201-Projet}{https://github.com/programindfr/INF201-Projet}

	\tableofcontents



\section{Scrutin uninominal}

\subsection{Question 1}

\begin{minted}[frame=single]{ocaml}
type candidat = string;; (* Réstriction aux chaînes représentants un candidat. *)
type bulletin = candidat;;
type urne = bulletin list;;
type score = int;; (* Réstriction aux entiers positifs. *)
type panel = candidat list;;
\end{minted}



\subsection{Question 2}

\paragraph{Spécification}

\begin{center}
	\begin{tabular}{|p{2cm}|p{\dimexpr\linewidth-2cm-4\tabcolsep-3\arrayrulewidth}|}
		\hline
		Sémantique
		&
		\mintinline{ocaml}{compte c u} renvoie le score d'un candidat \mintinline{ocaml}{c} pour une urne \mintinline{ocaml}{u}. %La fonction est récursive et utilise le pattern matching.
%
		\\
		\hline
		Profile
		&
		\mintinline{ocaml}{compte: candidat -> urne -> score}
%
		\\
		\hline
		Exemple
		&
		\begin{minted}{ocaml}
compte "Eric" ["Eric"; "Eric"; "Mathias"] = 2
compte "Mathias" ["Eric"; "Eric"] = 0
		\end{minted}
%
		\\
		\hline
	\end{tabular}
\end{center}

\paragraph{Implémentation}

\begin{minted}[frame=single]{ocaml}
let rec compte (c: candidat) (u: urne) : score =
	match u with
	| [] -> 0
	| t::q -> if t=c then 1+(compte c q) else (compte c q);;
\end{minted}



\subsection{Question 3}

\paragraph{Spécification}

\begin{center}
	\begin{tabular}{|p{2cm}|p{\dimexpr\linewidth-2cm-4\tabcolsep-3\arrayrulewidth}|}
		\hline
		Sémantique
		&
		\mintinline{ocaml}{depouiller lc u} renvoie la liste des scores par candidat à partir d'un panel \mintinline{ocaml}{lc} et d'une urne \mintinline{ocaml}{u}. %La fonction est récursive et utilise le pattern matching. Le résultat est une liste de couples \mintinline{ocaml}{candidat*score} de type \mintinline{ocaml}{resultat}.
%
		\\
		\hline
		Profile
		&
		\mintinline{ocaml}{depouiller: panel -> urne -> resultat list}
%
		\\
		\hline
		Exemple
		&
		\begin{minted}{ocaml}
depouiller ["Eric"; "Mathias"] ["Eric"; "Eric"; "Mathias"] = [("Eric", 2); ("Mathias", 1)]
depouiller ["Eric"; "Mathias"] ["Eric"; "Eric"] = [("Eric", 2); ("Mathias", 0)]
		\end{minted}
%
		\\
		\hline
	\end{tabular}
\end{center}

\paragraph{Implémentation}

\begin{minted}[frame=single]{ocaml}
type resultat = candidat*score;;
let rec depouiller (lc: panel) (u: urne) : resultat list =
	match lc with
	| [] -> []
	| t::q -> (t, compte t u)::(depouiller q u);;
\end{minted}



\subsection{Question 4}

\paragraph{Spécification}

\begin{center}
	\begin{tabular}{|p{2cm}|p{\dimexpr\linewidth-2cm-4\tabcolsep-3\arrayrulewidth}|}
		\hline
		Sémantique
		&
		\mintinline{ocaml}{union r1 r2} renvoie l'union de deux résultats \mintinline{ocaml}{r1} et \mintinline{ocaml}{r2} pour un même candidat. La fonction renvoie une erreur si les candidats sont différents. %Le résultat est un couple \mintinline{ocaml}{candidat*score} de type \mintinline{ocaml}{resultat}.
%
		\\
		\hline
		Profile
		&
		\mintinline{ocaml}{union: resultat -> resultat -> resultat}
%
		\\
		\hline
		Exemple
		&
		\begin{minted}{ocaml}
union ("Eric", 2) ("Eric", 7) = ("Eric", 9)
union ("Eric", 7) ("Mathias", 2) = "Les candidats sont différents."
		\end{minted}
%
		\\
		\hline
	\end{tabular}
\end{center}

\paragraph{Implémentation}

\begin{minted}[frame=single]{ocaml}
let union (r1: resultat) (r2: resultat) : resultat =
	let (c1, s1)=r1 and (c2, s2)=r2 in
	if c1=c2 then (c1, s1+s2) else failwith "Les candidats sont différents.";;
\end{minted}



\subsection{Question 5}

\paragraph{Spécification}

\begin{center}
	\begin{tabular}{|p{2cm}|p{\dimexpr\linewidth-2cm-4\tabcolsep-3\arrayrulewidth}|}
		\hline
		Sémantique
		&
		\mintinline{ocaml}{max_depouiller l} renvoie le résultat du candidat qui possède le meilleur score parmi les éléments présents dans la liste de résultat \mintinline{ocaml}{l}. La fonction renvoie une erreur si la liste est vide. %La fonction est récursive et utilise le pattern matching. Le résultat est un couple \mintinline{ocaml}{candidat*score} de type \mintinline{ocaml}{resultat}.
%
		\\
		\hline
		Profile
		&
		\mintinline{ocaml}{max_depouiller: resultat list -> resultat}
%
		\\
		\hline
		Exemple
		&
		\begin{minted}{ocaml}
max_depouiller [("Eric", 7); ("Mathias", 2)] = ("Eric", 7)
max_depouiller [] = "Il n'y à pas de résultat."
		\end{minted}
%
		\\
		\hline
	\end{tabular}
\end{center}

\paragraph{Implémentation}

\begin{minted}[frame=single]{ocaml}
let rec max_depouiller (l: resultat list) : resultat =
	match l with
	| [] -> failwith "Il n'y à pas de résultat."
	| [e] -> e
	| t::q -> let (c1, s1)=t and (c2, s2)=(max_depouiller q) in
		if s1>s2 then t else (c2, s2);;
\end{minted}



\subsection{Question 6}

\paragraph{Spécification}

\begin{center}
	\begin{tabular}{|p{2cm}|p{\dimexpr\linewidth-2cm-4\tabcolsep-3\arrayrulewidth}|}
		\hline
		Sémantique
		&
		\mintinline{ocaml}{vainqueur_scrutin_uninominal u lc} renvoie le candidat qui possède le meilleur score parmi les bulletins d'une urne \mintinline{ocaml}{u} et les candidats d'un panel \mintinline{ocaml}{lc}. La fonction renvoie une erreur si le panel est vide. %La fonction utilise des fonctions définis précédemment.
%
		\\
		\hline
		Profile
		&
		\mintinline{ocaml}{vainqueur_scrutin_uninominal: urne -> panel -> candidat}
%
		\\
		\hline
		Exemple
		&
		\begin{minted}{ocaml}
vainqueur_scrutin_uninominal ["Eric"; "Eric"; "Mathias"] ["Eric"; "Mathias"] = "Eric"
vainqueur_scrutin_uninominal ["Eric"; "Eric"; "Mathias"] [] = "Il n'y à pas de résultat."
		\end{minted}
%
		\\
		\hline
	\end{tabular}
\end{center}

\paragraph{Implémentation}

\begin{minted}[frame=single]{ocaml}
let vainqueur_scrutin_uninominal (u: urne) (lc: panel) : candidat =
	let (c, s)=max_depouiller (depouiller lc u) in c;;
\end{minted}



\subsection{Question 7}

\paragraph{Spécification suppr\_elem}

\begin{center}
	\begin{tabular}{|p{2cm}|p{\dimexpr\linewidth-2cm-4\tabcolsep-3\arrayrulewidth}|}
		\hline
		Sémantique
		&
		\mintinline{ocaml}{suppr_elem l e} renvoie la liste \mintinline{ocaml}{l} sans la première occurrence de l'élément \mintinline{ocaml}{e}. %La fonction est récursive et utilise le pattern matching.
%
		\\
		\hline
		Profile
		&
		\mintinline{ocaml}{suppr_elem: 'a list -> 'a -> 'a list}
%
		\\
		\hline
		Exemple
		&
		\begin{minted}{ocaml}
suppr_elem ["Eric"; "Mathias"] "Mathias" = ["Eric"]
		\end{minted}
%
		\\
		\hline
	\end{tabular}
\end{center}

\paragraph{Implémentation suppr\_elem}

\begin{minted}[frame=single]{ocaml}
let rec suppr_elem (l: 'a list) (e: 'a) : 'a list =
	match l with
	| [] -> []
	| t::q -> if t=e then q else t::(suppr_elem q e);;
\end{minted}

\paragraph{Spécification deux\_premiers}

\begin{center}
	\begin{tabular}{|p{2cm}|p{\dimexpr\linewidth-2cm-4\tabcolsep-3\arrayrulewidth}|}
		\hline
		Sémantique
		&
		\mintinline{ocaml}{deux_premiers u lc} renvoie le couple de résultats des deux candidats qui possèdent le meilleur score parmi les bulletins d'une urne \mintinline{ocaml}{u} et les candidats d'un panel \mintinline{ocaml}{lc}. La fonction renvoie une erreur si le panel est vide. %La fonction utilise des fonctions définis précédemment.
%
		\\
		\hline
		Profile
		&
		\mintinline{ocaml}{deux_premiers: urne -> panel -> resultat*resultat}
%
		\\
		\hline
		Exemple
		&
		\begin{minted}{ocaml}
deux_premiers ["Eric"; "Eric"; "Mathias"] ["Eric"; "Mathias"] = (("Eric", 2), ("Mathias", 1))
deux_premiers ["Eric"; "Eric"; "Mathias"] [] = "Il n'y à pas de résultat."
		\end{minted}
%
		\\
		\hline
	\end{tabular}
\end{center}

\paragraph{Implémentation deux\_premiers}

\begin{minted}[frame=single]{ocaml}
let deux_premiers (u: urne) (lc: panel) : resultat*resultat =
	let (c1, s1)=(max_depouiller (depouiller lc u)) in
		let c2=(max_depouiller (depouiller (suppr_elem lc c1) u)) in
	((c1, s1), c2);;
\end{minted}



\subsection{Question 8}
Bien qu'après le changement, Eric reste en tête des votes, Stan, qui était en deuxième position se fait remplacer par Kyle, bien que Stan ait plus de voix que Kyle sans la présence de Keny. On retrouve cette situation dans l'introduction aux élections américaines de 2000 et aux élections présidentielles françaises de 1988, 2002 et 2007. Mélanchon en 2022



\subsection{Question 9}
Dans ce système, les électeurs peuvent se retrouver avec plusieurs représentants qu'ils souhaiterais voter pour, mais ne peuvent choisir qu'un. Cela donne des situations où aucun de ces représentant a suffisamment de votes car ils sont répartis entre eux.



\section{Jugement majoritaire}

\subsection{Question 10}

Dans le cadre d'une élection avec 12 candidats il y a $ 6^{12} = 2176782336 $ bulletins possibles que l'on peut mettre dans l'urne. Dans le cadre d’une élection avec 13 possibilités, il y a $ 6^{13} = 13060694016 $ bulletins possibles que l’on peut mettre dans l’urne. Avec les choix à dispositions, le nombre de bulletins possible est multiplié par 6 à chaque candidat additionnel. Cela peut sembler excessif et pourrait rendre difficile le traitement des votes.



\subsection{Question 11}

\begin{minted}[frame=single]{ocaml}
type mention = Arejeter | Insuffisant | Passable | Assezbien | Bien | Tresbien;;
type bulletin_jm = mention list;;
type urne_jm = bulletin_jm list;;
\end{minted}



\subsection{Question 12}

\begin{minted}[frame=single]{ocaml}
let rec depouille_jm (u: urne_jm) : mention list list =
	match u with
	| [] -> []
	| e -> (List.map (List.hd) u)::(
		match (List.hd u) with
		| [] -> []
		| [e] -> []
		| e -> depouille_jm (List.map (List.tl) u)
	);;
\end{minted}



\subsection{Question 13}

\begin{minted}[frame=single]{ocaml}
let tri (l:'a list):'a list = List.sort compare l;;
let tri_mentions (u: mention list list) : mention list list = List.map (tri) u;;
\end{minted}



\subsection{Question 14}

\begin{minted}[frame=single]{ocaml}
let mediane (l: 'a list) : 'a = List.nth l ((List.length l) / 2);;
\end{minted}



\subsection{Question 15}

\begin{minted}[frame=single]{ocaml}
let meilleur_mediane (u: mention list list) : mention =
	let medList = List.map (mediane) (tri_mentions u) in
		tri medList |> List.rev |> List.hd;;
\end{minted}



\subsection{Question 16}

\begin{minted}[frame=single]{ocaml}
let supprime_perdants (u: mention list list) : mention list list =
	let med = meilleur_mediane u in
		List.map (
			fun (l: mention list ) : mention list -> if mediane l < med then [] else l
		) u;;
\end{minted}



\subsection{Question 17}

\begin{minted}[frame=single]{ocaml}
let rec supprime_mention (l: mention list) (e: mention) : mention list =
	match l with
	| [] -> []
	| t::q -> if t = e then q else t::(supprime_mention q e);;

let supprime_meilleure_mediane (u: mention list list) : mention list list =
	List.map (
		fun (l: mention list) : mention list ->
			if l = [] then [] else l |> mediane |> supprime_mention l
	) u;;
\end{minted}



\subsection{Question 18}

La fonction \mintinline{ocaml}{vainqueur_jm p ms} applique la fonction \mintinline{ocaml}{supprime_perdants u} tant que tous les candidats n'ont pas tous la même médiane. Ensuite la fonction \mintinline{ocaml}{supprime_meilleure_mediane u} est appliquée jusqu'au moment ou il ne reste plus qu'un seul candidat ou que la médiane d'un candidat soit supérieurs à toutes les autres.

\begin{minted}[frame=single]{ocaml}
let rec vainqueur_jm (p: panel) (ms: urne_jm) : candidat =
	let n = supprime_meilleure_mediane ms in
	if List.for_all (fun l -> l = []) n then
		let ln = List.length ms in
		let rec fw = function
			| [] -> ""
			| hd :: tl -> (
				match hd with
				| [] -> fw tl
				| [ _ ] -> List.length (hd :: tl) - ln |> List.nth p
				| _ -> assert false)
		in
		fw ms
	else vainqueur_jm p n
\end{minted}



\subsection{Question 19}




\subsection{Question 20}




\section{Recomptons les voix}

\subsection{Question 21}

\begin{minted}[frame=single]{ocaml}
type ville = string;;
type zone = Dpt of string | Reg of string;;
type arbre =
	| N of zone * arbre list
	| Bv of ville * (string * int) list;;
\end{minted}



\subsection{Question 22}




\subsection{Question 23}




\section{Conclusion}

\end{document}